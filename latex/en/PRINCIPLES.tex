\documentclass{article}
\usepackage[utf8]{inputenc}
\usepackage[english]{babel}
\usepackage{geometry}
\usepackage{enumitem}

\geometry{a4paper, margin=1in}

\title{Principles of Sober Dialogue}
\author{}
\date{}

\begin{document}

\maketitle

\section{Boundaries of Freedom}
Your freedom ends where another person's freedom begins.

\section{Freedom and Responsibility}
Freedom of speech does not mean freedom from responsibility and does not equal absolute anarchy.

\section{Subjectivity of Humor}
What seems funny to you is not necessarily funny to others.

\section{Uniqueness of Experience}
If you have a certain position, quality, opinion, or type of thinking, it doesn't mean everyone else has it.

\section{Self-Check Before Publishing}
Before any message or comment, think:
\begin{itemize}
    \item ``Does this violate the law?''
    \item ``How would I react if this were written to me?''
\end{itemize}

\section{Culture of Apologies}
Apologizing is not shame or weakness, but a healthy practice that increases authority.

\section{Explaining Criticism and Defense}
When criticizing or defending something, explain soberly and in detail.

\vspace{1cm}
\hrule
\vspace{0.5cm}

\section*{What ``Sober'' Means}

\textbf{A sober approach to dialogue includes:}

\begin{itemize}[label=--]
    \item \textbf{Without personal attacks}
    \item \textbf{Based on facts, not emotions}
    \item \textbf{With acknowledgment of the limitations of one's knowledge}
    \item \textbf{With readiness to change opinion when presented with new arguments}
\end{itemize}

\end{document}