\documentclass{article}
\usepackage[utf8]{inputenc}
\usepackage[russian]{babel}
\usepackage{geometry}
\usepackage{enumitem}
\geometry{a4paper, margin=1in}
\title{Принципы трезвого диалога}
\author{}
\date{}
\begin{document}
\maketitle
\section{Границы свободы}
Ваша свобода заканчивается там, где начинается свобода другого человека.
\section{Свобода и ответственность}
Свобода слова не означает свободу от ответственности и не равна абсолютной анархии.
\section{Субъективность юмора}
То, что кажется смешным вам, не обязательно смешно для других.
\section{Уникальность опыта}
Если у вас есть определённая позиция, качество, мнение или тип мышления, это не означает, что он есть у всех остальных.
\section{Самопроверка перед публикацией}
Перед любым сообщением или комментарием подумайте:
\begin{itemize}
    \item «Нарушает ли это закон?»
    \item «Как бы я отреагировал, если бы это написали мне?»
\end{itemize}
\section{Культура извинений}
Извинение — это не стыд или слабость, а здоровая практика, которая повышает авторитет.
\section{Объяснение критики и защиты}
Критикуя или защищая что-либо, объясняйте трезво и подробно.
\vspace{1cm}
\hrule
\vspace{0.5cm}
\section*{Что означает «трезвый»}
\textbf{Трезвый подход к диалогу включает:}
\begin{itemize}[label=--]
    \item \textbf{Без личных нападок}
    \item \textbf{Основанный на фактах, а не на эмоциях}
    \item \textbf{С признанием ограниченности своих знаний}
    \item \textbf{С готовностью изменить мнение при появлении новых аргументов}
\end{itemize}
\end{document}